\usepackage{luatexja}
\usepackage[hiragino-pron, nfssonly, deluxe, expert]{luatexja-preset}
\usepackage{pgfpages}
\usepackage{fontspec}
\usepackage{epigraph}
\usepackage{etoolbox}
\usepackage{tikz}
\usepackage{framed}
\usepackage{amsmath}
\usepackage{mathtools}
\usepackage{listings}
\usepackage{libertine}
\usepackage{bxcoloremoji}
\usepackage{xcolor}
\usepackage{multirow}
\usepackage{diagbox}
\usepackage{caption}
\usepackage{xsavebox}

\definecolor{links}{HTML}{2A1B81}
\hypersetup{colorlinks,linkcolor=,urlcolor=links}

\usetheme{Boadilla}
\usecolortheme{seahorse}
% \usefonttheme{serif}

\setbeamercolor{page number in head/foot}{bg=blue!10}
\makeatother
\setbeamertemplate{footline}
{
  \leavevmode%
  \hbox{%
    \begin{beamercolorbox}[wd=.4\paperwidth,ht=2.25ex,dp=1ex,center]{author in head/foot}%
      \usebeamerfont{author in head/foot}\insertshortauthor\hspace*{1ex}(\insertshortinstitute)
    \end{beamercolorbox}%
    \begin{beamercolorbox}[wd=.3\paperwidth,ht=2.25ex,dp=1ex,center]{title in head/foot}%
      \usebeamerfont{title in head/foot}\insertshorttitle
    \end{beamercolorbox}%
    \begin{beamercolorbox}[wd=.2\paperwidth,ht=2.25ex,dp=1ex,center]{date in head/foot}%
      \insertshortdate
    \end{beamercolorbox}%
    \begin{beamercolorbox}[wd=.1\paperwidth,ht=2.25ex,dp=1ex,center]{page number in head/foot}%
      \insertframenumber{} / \inserttotalframenumber\hspace*{1ex}
    \end{beamercolorbox}}%
  \vskip0pt%
}
\makeatletter

\beamertemplatenavigationsymbolsempty

\setbeamertemplate{bibliography item}{\insertbiblabel}
\setbeamersize{description width=1cm}
\setbeamertemplate{items}[circle]
\setbeamertemplate{section in toc}[circle]
\setbeamertemplate{subsection in toc}{%
  \leavevmode\leftskip=2em
  {%
    \usebeamerfont*{itemize item}%
    \usebeamercolor{subsection number projected}%
    \color{bg}%
    \raise1.25pt\hbox{\donotcoloroutermaths$\bullet$}}%
  \hskip1.5ex\inserttocsubsection\par}
\setbeamercolor{title}{bg=white}
\setbeamertemplate{title page}
{%
  \vbox{}
  \vfill
  \begingroup
    \centering
    \hrulefill
    \vskip1em\par
    \begin{beamercolorbox}[sep=8pt,center,shadow=false,rounded=true]{title}
      \usebeamerfont{title}\inserttitle\par%
      \ifx\insertsubtitle\@empty%
      \else%
        \vskip0.25em%
        {\usebeamerfont{subtitle}\usebeamercolor[fg]{subtitle}\insertsubtitle\par}%
      \fi%     
    \end{beamercolorbox}%
    \hrulefill
    \vskip1em\par
    \begin{beamercolorbox}[sep=8pt,center,shadow=false,rounded=true]{author}
      \usebeamerfont{author}\insertauthor
    \end{beamercolorbox}
    \begin{beamercolorbox}[sep=8pt,center,shadow=false,rounded=true]{institute}
      \usebeamerfont{institute}\insertinstitute
    \end{beamercolorbox}
    \begin{beamercolorbox}[sep=8pt,center,shadow=false,rounded=true]{date}
      \usebeamerfont{date}\insertdate
    \end{beamercolorbox}\vskip0.5em
    {\usebeamercolor[fg]{titlegraphic}\inserttitlegraphic\par}
  \endgroup
  \vfill
}
\setbeamertemplate{blocks}[rounded][shadow=false]
\setbeamertemplate{note page}{\pagecolor{yellow!5}\insertnote}


% ============ ここを消すとNote消える ================
\mode<handout>{%
  \setbeameroption{show notes on second screen=right}%
}
% ============ ここを消すとNote消える ================


\renewcommand{\kanjifamilydefault}{\gtdefault}

\resetcounteronoverlays{lstlisting}
\definecolor{bluegray}{rgb}{0.4, 0.6, 0.8}
\DeclareCaptionFormat{listing}{{\color{bluegray}\lstlistingname}#2#3}
\captionsetup[lstlisting]{format=listing, font={footnotesize}}

\setmonofont[Ligatures=TeX]{CMU Typewriter Text}

\newcommand{\xcolon}{:}

\title[Generics in protocol/extension]{%
  {\bfseries\rmfamily\mcfamily\huge
    Generics in protocol/extension%
  }%
}
\author[Hikaru Yoshimura]{%
  Hikaru \textsc{Yoshimura}(吉村 優)
}
\date[try! Swift Tokyo, March 21, 2019]{%
  \href{https://www.tryswift.co/events/2019/tokyo/en/}{try! Swift Tokyo on \oldstylenums{March 21, 2019}} \\
  {\scriptsize (\href{https://github.com/y-yu/try-swift-slide}{\texttt{y-yu/try-swift-slide@\GITAbrHash}})}%
}
\institute[Recruit Markting Partners Co., Ltd.]{%
  Recruit Markting Partners Co., Ltd.\\
  \href{mailto:yyu@mental.poker}{yyu@mental.poker}
}

\newfontfamily\quotefont[Ligatures=TeX]{Linux Libertine O} % selects Libertine as the quote font

\newcommand*\quotesize{60} % if quote size changes, need a way to make shifts relative
% Make commands for the quotes
\newcommand*{\openquote}
   {\tikz[remember picture,overlay,xshift=-4ex,yshift=-2.5ex]
   \node (OQ) {\quotefont\fontsize{\quotesize}{\quotesize}\selectfont``};\kern0pt}

\newcommand*{\closequote}[1]
  {\tikz[remember picture,overlay,xshift=1.5ex,yshift={#1}]
   \node (CQ) {\quotefont\fontsize{\quotesize}{\quotesize}\selectfont''};}

\newcommand*\shadedauthorformat{\emph} % define format for the author argument

% Now a command to allow left, right and centre alignment of the author
\newcommand*\authoralign[1]{%
  \if#1l
    \def\authorfill{}\def\quotefill{\hfill}
  \else
    \if#1r
      \def\authorfill{\hfill}\def\quotefill{}
    \else
      \if#1c
        \gdef\authorfill{\hfill}\def\quotefill{\hfill}
      \else\typeout{Invalid option}
      \fi
    \fi
  \fi}
% wrap everything in its own environment which takes one argument (author) and one optional argument
% specifying the alignment [l, r or c]
%
\newenvironment{shadequote}[2][l]%
{\authoralign{#1}
\ifblank{#2}
   {\def\shadequoteauthor{}\def\yshift{-2ex}\def\quotefill{\hfill}}
   {\def\shadequoteauthor{\par\authorfill\shadedauthorformat{#2}}\def\yshift{2ex}}
\begin{quote}\normalfont\openquote}
{\shadequoteauthor\quotefill\closequote{\yshift}\end{quote}}

\makeatletter
\def\@fnsymbol#1{\ensuremath{\ifcase#1\or *\or \dagger\or \ddagger\or
   \mathsection\or \mathparagraph\or \|\or **\or \dagger\dagger
   \or \ddagger\ddagger \else\@ctrerr\fi}}
\makeatother

\renewcommand{\thefootnote}{\fnsymbol{footnote}}

\usetikzlibrary{shapes.callouts} 

\pgfkeys{%
    /calloutquote/.cd,
    width/.code                   =  {\def\calloutquotewidth{#1}},
    position/.code                =  {\def\calloutquotepos{#1}}, 
    author/.code                  =  {\def\calloutquoteauthor{#1}},
    at/.code                      =  {\def\calloutquoteat{#1}},
    /calloutquote/.unknown/.code   =  {\let\searchname=\pgfkeyscurrentname
                                 \pgfkeysalso{\searchname/.try=#1,                                
    /tikz/\searchname/.retry=#1},\pgfkeysalso{\searchname/.try=#1,
                                  /pgf/\searchname/.retry=#1}}
                            }  


\newcommand\calloutquote[2][]{%
       \pgfkeys{/calloutquote/.cd,
         width               = 5cm,
         position            = {(0,-1)},
         at                  = {(0,0)},
         author              = {}}
  \pgfqkeys{/calloutquote}{#1}                   
  \node [rectangle callout,callout relative pointer={\calloutquotepos},text width=\calloutquotewidth,/calloutquote/.cd,
     #1] (tmpcall) at \calloutquoteat {\hfil#2\hfil};
  \node at (tmpcall.pointer){\calloutquoteauthor};    
}

\newfontfamily\listingfont{Menlo}
\definecolor{dkgreen}{rgb}{0,0.6,0}
\definecolor{gray}{rgb}{0.5,0.5,0.5}
\definecolor{mauve}{rgb}{0.58,0,0.82}

\makeatletter
\lst@CCPutMacro\lst@ProcessOther {"2D}{\lst@ttfamily{-{}}{-{}}}
\@empty\z@\@empty
\makeatother

\lstdefinelanguage{swift}{
  morekeywords={
    func,if,then,else,for,in,while,do,switch,case,default,where,break,continue,fallthrough,return,
    typealias,struct,class,enum,protocol,var,func,let,get,set,willSet,didSet,inout,init,deinit,extension,
    subscript,prefix,operator,infix,postfix,precedence,associativity,left,right,none,convenience,dynamic,
    final,lazy,mutating,nonmutating,optional,override,required,static,unowned,safe,weak,internal,
    private,public,is,as,self,unsafe,dynamicType,true,false,nil,Type,Protocol,import
  },
  morecomment=[l]{//},
  morecomment=[s]{/*}{*/},
  morestring=[b]"
}

\lstset{
  basicstyle=\listingfont\small,
  frame=single,
  xleftmargin=2em,
  xrightmargin=1em,
  breaklines=true
}

\lstdefinestyle{swift}{
  basicstyle=\listingfont\scriptsize,
  breakatwhitespace=false,
  language=swift,
  captionpos=b,
  commentstyle=\listingfont\scriptsize\color{dkgreen},
  extendedchars=true,
  xleftmargin=1em,
  xrightmargin=0.5em,
  keepspaces=true,
  keywordstyle=\listingfont\scriptsize\color{blue},
  emphstyle=\listingfont\scriptsize\color{cyan},
  rulecolor=\listingfont\scriptsize\color{black},
  showspaces=false,
  showstringspaces=false,
  showtabs=false,
  stringstyle=\listingfont\scriptsize\color{mauve},
  tabsize=2
}

\lstdefinestyle{java}{
  basicstyle=\listingfont\scriptsize,
  breakatwhitespace=false,
  language=Java,
  captionpos=b,
  commentstyle=\listingfont\scriptsize\color{dkgreen},
  extendedchars=true,
  xleftmargin=1em,
  xrightmargin=0.5em,
  keepspaces=true,
  keywordstyle=\listingfont\scriptsize\color{blue},
  emphstyle=\listingfont\scriptsize\color{cyan},
  rulecolor=\listingfont\scriptsize\color{black},
  showspaces=false,
  showstringspaces=false,
  showtabs=false,
  stringstyle=\listingfont\scriptsize\color{mauve},
  tabsize=2
}

\lstdefinestyle{csharp}{
  numbers=left,
  language=[Sharp]C
}

\lstdefinestyle{cil}{
  numbers=left,
  language=CIL
}

\lstdefinestyle{plain}{
  numbers=left,
  language=
}

\lstdefinestyle{sh}{
  numbers=left,
  language=sh
}

\lstdefinestyle{c}{
  numbers=left,
  language=C
}

\lstdefinestyle{python}{
  numbers=left,
  language=Python
}

\lstdefinestyle{asm-x86}{
  numbers=left
}

\lstdefinestyle{pseudo-code}{
  numbers=left,
  keywords=[6]{for,from,to,endfor,while,endwhile}
}

\lstdefinestyle{bitcoin-script}{
  mathescape=true
}

\lstdefinestyle{scala}{
  basicstyle=\listingfont\scriptsize,
  breakatwhitespace=false,
  language=scala,
  captionpos=b,
  commentstyle=\listingfont\scriptsize\color{dkgreen},
  extendedchars=true,
  xleftmargin=1em,
  xrightmargin=1em,
  keepspaces=true,
  keywordstyle=\listingfont\scriptsize\color{blue},
  emphstyle=\listingfont\scriptsize\color{cyan},
  rulecolor=\listingfont\scriptsize\color{black},
  showspaces=false,
  showstringspaces=false,
  showtabs=false,
  stringstyle=\listingfont\scriptsize\color{mauve},
  tabsize=2
}

\lstdefinelanguage{scala}{
  morekeywords={abstract,case,catch,class,def,%
    do,else,extends,false,final,finally,%
    for,if,implicit,import,match,mixin,%
    new,null,object,override,package,%
    private,protected,requires,return,sealed,%
    super,this,throw,trait,true,try,%
    type,val,var,while,with,yield},
  moreemph={Byte,Short,Int,Long,Float,Double,Char,
    String,Boolean,Unit,Null,Nothing,Any,AnyRef,
    Left,Right,Either},
  otherkeywords={=>,<-,<\%,<:,>:,\#,@},
  sensitive=true,
  morecomment=[l]{//},
  morecomment=[n]{/*}{*/},
  morestring=[b]",
  morestring=[b]',
  morestring=[b]"""
}
\def\AtSOne#1\csod{%
	\begin{array}{c|}
		\hline
		#1\\
		\hline
	\end{array}
}%
\def\AtSTwo#1,#2\csod{%
	\begin{array}{c|c|}
		\hline
		#1 & #2\\
		\hline
	\end{array}
}%
\def\AtSThree#1,#2,#3\csod{%
	\begin{array}{c|c|c|}
		\hline
		#1 & #2 & #3\\
		\hline
	\end{array}
}%
\def\AtSFour#1,#2,#3,#4\csod{%
	\begin{array}{c|c|c|c|}
		\hline
		#1 & #2 & #3 & #4\\
		\hline
	\end{array}
}%
\def\AtSFive#1,#2,#3,#4,#5\csod{%
	\begin{array}{c|c|c|c|c|}
		\hline
		#1 & #2 & #3 & #4 & #5\\
		\hline
	\end{array}
}%
\def\AtSSix#1,#2,#3,#4,#5,#6\csod{%
	\begin{array}{c|c|c|c|c|c|}
		\hline
		#1 & #2 & #3 & #4 & #5 & #6\\
		\hline
	\end{array}
}
\newcommand{\SOne}[1]{\AtSOne#1\csod}
\newcommand{\STwo}[1]{\AtSTwo#1\csod}
\newcommand{\SThree}[1]{\AtSThree#1\csod}
\newcommand{\SFour}[1]{\AtSFour#1\csod}
\newcommand{\SFive}[1]{\AtSFive#1\csod}
\newcommand{\SSix}[1]{\AtSSix#1\csod}
\input{vc.tex}

\setbeamertemplate{items}[circle]

\newcommand\ballcircle[1]{%
  {%
    \usebeamercolor{enumerate item}%
    \tikzset{beameritem/.style={circle,inner sep=0,minimum size=2ex,text=enumerate item.bg,fill=enumerate item.fg,font=\footnotesize}}%
    \tikz[baseline=(n.base)]\node(n)[beameritem]{#1};%
  }
}
\newcommand\ballref[1]{%
  \ballcircle{\ref{#1}}
}

\newcommand\ce[1]{%
  \coloremoji{#1}
}

\newenvironment{notes}
  {%
    \begin{xlrbox}{NotesBox}
    \begin{minipage}{.95\textwidth}
    \small\rmfamily\mcfamily
    \begin{itemize}
    \setlength{\itemindent}{0em}
  }{%
    \end{itemize}
    \end{minipage}
    \end{xlrbox}
    \note{\theNotesBox}}

\makeatletter
\newsavebox\temp@simple@callout@box
\newcommand{\simplecallout}[3]{%
  \sbox{\temp@simple@callout@box}{\mbox{#3}}%
  \begin{center}%
    \begin{tikzpicture}%
      \calloutquote[width=1.1\wd\temp@simple@callout@box,position={(#1.5,-0.2)},fill=#2,rounded corners]{
        #3%
      }%
    \end{tikzpicture}%
  \end{center}
}
\makeatother
  
\begin{document}

\begin{frame}
  \vspace{5ex}
  \centering
  \maketitle
  
  \begin{notes}
    \item \textcolor{red}{この右側のページはコメントが書いてあります}
    (最終的には消えます)

    \item スライドが16:9になりました!

    \item 基本的にスライドは英語で、発表は日本語です

    \item 右側に英語の用語の日本語訳も書きます

    \item ちょっと文字が多すぎるかもしれない。
    ただスライドにないことを喋るのは大変だけど、
    あることを当日に飛ばすのはまだ容易なので、
    つくれるだけ作って練習しつつどうにかしようかな

    \item Transcript:
    このトークでは「どうやったらGenericsをprotocol/extension」で利用できるか?
    についてHListというデータ構造の結合を例として解説する。
    まずは問題について詳しく述べて、そしてSwiftのprotocol/extensionについて
    Javaのインターフェースと比較しながら述べていく。
    そして実はprotocol/extensionはオーバーロードのような機能があることを
    利用して問題をなんとか解決する。
    最後に現在のSwiftでは難しかったところについて述べて
    参考文献を紹介する。
  \end{notes}
\end{frame}

\begin{frame}
  \frametitle{Table of Contents}

  \tableofcontents

  \begin{notes}
    \item Transcript:
    今日の流れ
  \end{notes}
\end{frame}

\section{Self-introduction}
\begin{frame}
  \frametitle{Self-introduction}
  
  \begin{columns}
    \begin{column}{0.3\textwidth}
      \begin{center}
        \begin{figure}
          \includegraphics[width=0.85\textwidth]{img/face.jpg}
        \end{figure}
      \end{center}
 
      \begin{table}[h]
        \begin{tabular}{ll}
          Twitter & \href{https://twitter.com/\_yyu\_}{@\_yyu\_} \\
          Qiita & \href{https://qiita.com/yyu}{yyu} \\
          GitHub & \href{https://github.com/y-yu}{y-yu} \\
        \end{tabular}
      \end{table}
    \end{column}
    \begin{column}{0.7\textwidth}
      \pause
      \begin{itemize}
        \item University of Tsukuba (Bachelor of CS)
        \begin{itemize}
          \item Programming Logic Group
        \end{itemize}
        
        \item Scala engineer at
        \href{http://www.recruit-mp.co.jp/}{Recruit Marketing Partners Co., Ltd.}

        \item I'm interested in
        \begin{itemize}
          \item Cryptography
          \item Functional Programming
          \item Quantum Information
          \item Typesetting (\LaTeX)
        \end{itemize}
      \end{itemize}
    \end{column}
  \end{columns}

  \begin{notes}
    \item 用語
    \begin{description}
      \item[Logic] 論理
      \item[Scala] スカラ
      \item[Cryptography] 暗号
      \item[Functional Programming] 関数プログラミング
      \item[Quantum Information] 量子情報
      \item[\LaTeX] ラテック・ラテフ(欧米圏ではラテックらしい)
    \end{description}

    \item もし発表で「量子」と言ったら、それは``quantum''でおねがいしたいです。
    たとえば「量子コンピュータ」は``quantum computer''とかです

    \item Transcript:
    筑波大学でプログラム・ロジックグループという研究室に所属し、
    型システムの研究を行った。そして現在はリクルートマーケティングパートナーズにて
    Scalaのサーバーサイドエンジニアをしている。
    業務では関数型プログラミングをしている。
    また、暗号や量子コンピュータにも興味がある。
    そしてこのスライドを作るためにも利用した
    組版ソフト\LaTeX を使って本を作成するといったこともしている。
  \end{notes}
\end{frame}

\section{Problem}

\begin{frame}[fragile]
  \frametitle{Problem}

  \begin{itemize}
    \item First we are thinking about these data structures: \lstinline|Tuple| and \lstinline|List|
  \end{itemize}
  
  \begin{columns}
    \begin{column}{0.5\textwidth}
\begin{lstlisting}[style=swift]
struct Tuple<A, B> {
  let left: A
  let right: B
 
  init(_ l: A, _ r: B) {
    this.left = l
    this.right = r
  }
}
\end{lstlisting}
    \end{column}
    \begin{column}{0.5\textwidth}
\begin{lstlisting}[style=swift]
enum List<A> {
  indirect case Cons(
    h: A,
    t: List<A>
  )
  case Nil
}
\end{lstlisting}
    \end{column}
  \end{columns}

  \pause
  \begin{itemize}
    \item<+-> \lstinline|Tuple| represents \emph{exact two} values
    for each type \lstinline|A| and \lstinline|B|

    \item<+-> And \lstinline|List| represents fixed type \lstinline|A| values
    whose number is \emph{more than or equal 0}
  \end{itemize}

  \uncover<+->{
    \simplecallout{-}{blue!20}{Can we make a List which have some type values?}
  }

  \begin{notes}
    \item 用語
    \begin{description}
      \item[data structure] データ造構
      \item[Tuple] タプル(組)
      \item[List] リスト
      \item[type] 型
      \item[Cons] コンス(専門用語)
      \item[Nil] ニル(専門用語)
    \end{description}

    \item Transcript:
    タプルはちょうど1つづつAとBの型を持つ値を表し、
    そしてリストは1つの型Aをもつ0個以上の値を表す。
    ただ、我々は違う型の値が何個でもはいる型をつくれないだろうか?
  \end{notes}
\end{frame}

\begin{frame}[fragile]
  \frametitle{Heterogeneous List}

  \begin{itemize}
    \item We are looking at \lstinline|HList| as follows
  \end{itemize}

\begin{lstlisting}[style=swift]
protocol HList {
  associatedtype Head
  associatedtype Tail: HList
}
\end{lstlisting}

  \pause
  \begin{itemize}
    \item<+-> \lstinline|HList| has two \lstinline|associatedtype|: \lstinline|Head| and \lstinline|Tail|
  \end{itemize}

  \begin{notes}
    \item 用語
    \begin{description}
      \item[HList] Hリスト(Heterogeneous List)
      \item[type member] 型メンバー
    \end{description}
 
    \item HeadとTailも専門用語かも?

    \item Transcript:
    ここで今日は``HList''というデータ構造を紹介する。
    HListは``Head''と``Tail''という2つの型メンバー(associatedtype)を
    持つ。
  \end{notes}
\end{frame}

\begin{frame}[fragile]
  \frametitle{Heterogeneous List}

  \begin{itemize}
    \item<+-> And there are two structs: \lstinline|HCons| and \lstinline|HNil|
  \end{itemize}

  \begin{columns}
    \begin{column}{0.45\textwidth}
\begin{lstlisting}[style=swift]
public enum Nothing { }    
struct HNil: HList {
  typealias Head = Nothing
  typealias Tail = HNil
   
  init() { }
   
  static var hNil = HNil()
}   
\end{lstlisting}
    \end{column}
    \begin{column}{0.55\textwidth}
\begin{lstlisting}[style=swift]
struct HCons<H, T: HList>: HList {
  typealias Head = H
  typealias Tail = T
  let head: Head
  let tail: Tail
   
  init(_ h: Head, _ t: Tail) {
    self.head = h
    self.tail = t
  }
}
\end{lstlisting}
    \end{column}
  \end{columns}

  \uncover<+->{
    \simplecallout{}{red!30}{Is it the same to \lstinline|List|?}
  }
  \uncover<+->{
    \simplecallout{-}{green!30}{No, it's not just \lstinline|List|}
  }

  \begin{notes}
    \item 用語
    \begin{description}
      \item[(H)Cons] (H)コンス
      \item[struct] 構造体
    \end{description}

    \item Consは「コンス」という用語でOK

    \item Transcript:
    そして、次のような2つの構造体である``HCons''と``HNil''があり、
    これらはこの図のようになっている。
    これは一見するとリスト(List)のように見えるが、実は単なるリストではない。
  \end{notes}
\end{frame}

\begin{frame}[fragile]
  \frametitle{Heterogeneous List}

  \begin{itemize}
    \item<+-> \lstinline|HList| can have some values
    and their type order
  \end{itemize}

  \begin{columns}
    \begin{column}{0.3\textwidth}
      \begin{uncoverenv}<+->
\begin{lstlisting}[style=swift]
let a = HCons(
  1,
  HCons(
    true,
    HCons(
      "string",
      HNil.hnil() 
    )
  )
)
\end{lstlisting}
      \end{uncoverenv}
    \end{column}
    \begin{column}{0.7\textwidth}
      \begin{uncoverenv}<+->
\begin{lstlisting}[style=swift]
a: HCons<Int, HCons<Bool, HCons<String, HNil>>>
\end{lstlisting}
      \end{uncoverenv}
    \end{column}
  \end{columns}

  \begin{itemize}
    \item<+-> \lstinline|List| cannot have more than 1 type, but \lstinline|HList| have any of the number of types
  \end{itemize}

  \uncover<+->{
    \simplecallout{}{green!30}{Today, we'll consider the \lstinline|append| method of two \lstinline|HList|s}
  }

  \begin{notes}
    \item 型はまあいいかな?(Nilがちょっと心配?)

    \item ここでトークではHListのappendについて考えると導入する

    \item Transcript:
    HListはいくつかの値と、その値の型の順序を持つことができる。
    たとえば、左の図のようにIntの1、Boolのtrue、そして文字列の"string"が並んだような
    データ構造も表わすことができる。
    これはSwiftの現在の処理系でコンパイルすることができ、
    右図のような型がきちんとつく。
    Listは1つ以上の型を持つことはできないが、
    一方でHListは何種類でも型を持つことができる。
    今日は、2つのHList(のインスタンス)が与えられたときに、
    それらのappend(結合)を考えていきたい。
  \end{notes}
\end{frame}


\begin{frame}[fragile]
  \frametitle{\texttt{append} for \texttt{List}s}

  \begin{itemize}
    \item<+-> For now, in the case of the ordinary \lstinline|List|
  \end{itemize}

  \begin{uncoverenv}<+->
\begin{lstlisting}[style=swift]
func append<A>(_ a: List<A>, _ b: List<A>) -> List<A> {
  switch a {
  case let .Cons(h, t):
    return .Cons(h, append(t, b))
  case .Nil:
    return b
  }
}  
\end{lstlisting}     
  \end{uncoverenv}

  \begin{itemize}
    \item<+-> But, \lstinline|HList| may have more than 1 type,
    so it's not easy...\ce{😇}
  \end{itemize}

  \uncover<+->{
    \simplecallout{-}{blue!20}{What should we use to implement it?}
  }

  \uncover<+->{
    \simplecallout{}{green!30}{It's difficult for Java
      but Swift has \lstinline|protocol| and \lstinline|extension|}
  }

  \begin{notes}
    \item 用語
    \begin{description}
      \item[protocol] プロトコル
      \item[extension] エクステンション
    \end{description}
    
    \item appendのことをトーク中は``concat(concatination)''のように言うかも

    \item Transcript:
    まずは普通のListについて考える。
    普通のListの結合(append)はこのように再帰的(recursive)に実装できる。
    このListのappendは1つ型パラメータAを使っているが、
    しかしHListはいくつも型がある。
    こういうときどうやって実装したらよいだろうか?
    Javaでは難しいが、Swiftには``protocol/extension''があるので、
    これが達成できる。
  \end{notes}
\end{frame}

\section{protocol/extension}

\begin{frame}[fragile]
  \frametitle{\texttt{protocol}/\texttt{extension}}

  \pause
  \begin{itemize}
    \item<+-> Swift \lstinline|protocol| sometimes could be said
    ``it's similler to \lstinline|interface| of Java''
  \end{itemize}

  \begin{uncoverenv}<+->
    \begin{columns}
      \begin{column}{0.55\textwidth}
\begin{lstlisting}[style=java, caption={Java interface}]
interface JsonWrite<A> {
  public JSON toJson(A a);
}
class User(???) implements JsonWrite<User> {
  public JSON toJson(User a) {
    ???
  }
}
\end{lstlisting}
    \end{column}
    \begin{column}{0.45\textwidth}
\begin{lstlisting}[style=swift, caption={Swift protocol}]
protocol JsonWrite {
  associatedtype A
  func toJson(_ a: A) -> JSON
}
class User: JsonWrite {
  typealias A = User
  func toJson(_ a: A) -> JSON {
    ???
  }
}
\end{lstlisting}
      \end{column}
    \end{columns}
  \end{uncoverenv}

  \uncover<+->{
    \simplecallout{}{green!30}{It's not at all...}
  }

  \begin{notes}
    \item 用語
    \begin{description}
      \item[interface] インターフェース
      \item[JSON] JSON(ジェイソン)
      \item[Haskell] ハスケル
    \end{description}

    \item Transcript:
    ここでSwiftが持つprotocol/extensionについて考える。
    これは僕がSwiftを好きに思う理由の1つで、とてもおもしろい機能である。
    HaskellやScalaにも、このような機能があるがJavaにはないと思う。
    ただ、しばしば「SwiftのprotocolはJavaのinterfaceに似ている」と
    言われることがある。たしかに次のようなJSONに変換できるような
    protocolである``JsonWrite''を考えると、今``User''が
    ``JsonWrite''を実装する(implement)、または適合する(confirm)ことを
    JavaとSwiftでそれぞれ同じように書ける。
    これだけだと確かに似ているが、ただそれだけではない。
  \end{notes}
\end{frame}

\begin{frame}[fragile]
  \frametitle{\texttt{protocol}/\texttt{extension}}

  \begin{itemize}
    \item<+-> If we need to extend a fixed data type such a \lstinline|Integer|

    \item<+-> However in Java, we cannot extend a fixed data type so
    we can't help but to use the \emph{Adapter pattern}
  \end{itemize}

  \begin{columns}
    \begin{column}{0.6\textwidth}
      \begin{uncoverenv}<.->
\begin{lstlisting}[style=java, caption={Adapter pattern in Java}]
class RichInteger implements JsonWrite<Integer> {
  public Integer n;
  public RichInteger(Integer i) {
    this.n = i
  }
  
  public JSON toJson(Integer a) {
    ???
  }
}
\end{lstlisting}
      \end{uncoverenv}
    \end{column}
    \begin{column}{0.4\textwidth}
      \begin{uncoverenv}<+->
\begin{lstlisting}[style=swift, caption={\texttt{extension} in Swift}]
extension Int: JsonWrite {
  typealias A = Int

  func toJson(_ a: Int) -> JSON {
    ???
  }
}    
\end{lstlisting}
      \end{uncoverenv}
    \end{column}
  \end{columns}

  \begin{itemize}
    \item<.-> But Swift has \lstinline|extension|,
    which find the function by the given types
  \end{itemize}

  \begin{notes}
    \item 用語
    \begin{description}
      \item[Adapter pattern] アダプターパターン
    \end{description}

    \item Transcript:
    もし、すでに固まってしまったIntegerのようなデータ構造をJSONに変換
    したくなったとき、どうしたらよいだろうか。
    このような固定されたデータ構造は、たとえばプログラミング言語のコアや
    あるいは他人のライブラリーなどで定義された型であって、
    我々の努力で今つくったJsonWriteを使ってもらうのは不可能である。
    こういうとき、JavaではAdapter patternという手法で次のような
    RichIntegerのような型を新たにつくって対処することがしばしばある。
    これはこれで動作はするが、しかし我々の関心はIntegerから
    RichIntegerへ移動してしまった。これによりプログラムの複雑さが不必要に
    高まってしまう。一方でSwiftはextensionにより、この問題をスマートに解決できる。
    すでにフィックスされた型に対して、アドホックに機能を付け加えることが容易にできる。
    これはextensionが「型から機能を探してくる」という能力を持っているからである。
  \end{notes}
\end{frame}


\begin{frame}[fragile]
  \frametitle{\texttt{protocol}/\texttt{extension}}
  
  \uncover<+->{
    \simplecallout{}{red!30}{What is the meaning of finding functions by the types?}
  }

  \begin{itemize}
    \item<+-> For example, let's think about \emph{overloading} like following
  \end{itemize}

  \begin{uncoverenv}<.->
    \begin{columns}
      \begin{column}{0.5\textwidth}
\begin{lstlisting}[style=swift]
1 + 1 // 2
\end{lstlisting}
    \end{column}
    \begin{column}{0.5\textwidth}
\begin{lstlisting}[style=swift]
"hot" + "dog" // "hotdog"
\end{lstlisting}
      \end{column}
    \end{columns}
  \end{uncoverenv}

  \begin{itemize}
    \item<+-> Operator \lstinline|+| has two behaviors:
    \begin{itemize}
      \item \lstinline|(Int, Int) -> Int|
      \item \lstinline|(String, String) -> String|
    \end{itemize}

    \item<+-> So Swift looks up a suitable function by
    the arguments' types

    \item<+-> \lstinline|extension| is simillar like overloading
  \end{itemize}

  \begin{notes}
    \item 用語
    \begin{description}
      \item[overload] オーバーロード
      \item[operator] 演算子
      \item[argument] 引数
    \end{description}

    \item Transcript:
    「型から機能を探してくる」とはどういうことだろうか。
    これの簡単な例として``overload''がある。
    おなじ``+''という関数または演算子だが、
    引数によって左の例では加算となり、右の例では文字列の連結(concatination)となる。
    これは+に引数の型に応じて少なくとも2つの振る舞いがあり、
    もし引数の型がIntとIntならば足し算としてIntを返し、
    そしてもしStringとStringならば文字列の連結としてStringを返す。
    つまり、Swiftが適切な機能を引数の型から探してきてくれる。
    extensionはこれに似た振る舞いをする。
  \end{notes}
\end{frame}

\begin{frame}[fragile]
  \frametitle{\texttt{protocol}/\texttt{extension}}

  \begin{itemize}
    \item<+-> We can write \lstinline|+| function using
    \lstinline|protocol| and \lstinline|extension|
  \end{itemize}
  
  \begin{uncoverenv}<.->
\begin{lstlisting}[style=swift]
protocol Plusable {
  associatedtype A
  func +(_ a: A, _ b: A) -> A
}    
\end{lstlisting}

    \begin{columns}
      \begin{column}{0.5\textwidth}
\begin{lstlisting}[style=swift]
extension Int: Plusable {
  typealias A = Int
  func +(_ a: A, _ b: A) -> A {
    // `Int.plus` maybe not exist...
    return Int.plus(a, b)
  }
}
\end{lstlisting}
    \end{column}
    \begin{column}{0.5\textwidth}
\begin{lstlisting}[style=swift]
extension String: Plusable {
  typealias A = String
  func +(_ a: A, _ b: A) -> A {
    return a.stringByAppendingString(b)
  }
}
\end{lstlisting}
      \end{column}
    \end{columns}
  \end{uncoverenv}

  \begin{itemize}
    \item<+-> Overloading can be done by \lstinline|protocol| and \lstinline|extension|
  \end{itemize}

  \begin{notes}
    \item Transcript:
    オーバーロードのようなことができると示すために、
    この+をprotocol/extensionで書いてみる。
    ``Plusable''というprotocolをつくり、それのextensionとして
    IntとStringにそれぞれ+演算子を定義する。
    このように、オーバーロードがすることはprotocol/extensionで
    同じように行うことができる。
  \end{notes}
\end{frame}

\begin{frame}[fragile]
  \frametitle{\texttt{protocol}/\texttt{extension}}

  \begin{itemize}
    \item<+-> \lstinline|extension| can use another
    \lstinline|extension| defined by a \lstinline|protocol|
  \end{itemize}

  \begin{uncoverenv}<.->
\begin{lstlisting}[style=swift]
extension Array: JsonWrite where Element: JsonWrite {
  typealias A = Array<Element>

  func toJson(_ a: Array<Element>): JSON {
    ???
  }
}
\end{lstlisting}
  \end{uncoverenv}

  \uncover<+->{
    \simplecallout{}{green!30}{%
      Can we make the \lstinline|append| for \lstinline|HList|s
      by \lstinline|extension|?}
  }

  \begin{notes}
    \item Transcript:
    さらにextensionは他のprotocolで定義されたextensionを
    使うこともできる。
    たとえばここでは``Array''にさきほどのJsonWriteを適合させている。
    ただし``where''句をともなって、Arrayの型パラメータAが
    JsonWriteに適合しているならば、Array<A>もJsonWriteに
    適合するといったことが書ける。
    このextensionを使えばHListのappendを作れるのでは?
  \end{notes}
\end{frame}

\section{Implementation}

\begin{frame}[fragile]
  \frametitle{Implementation}

  \begin{itemize}
    \item<+-> This is a \lstinline|append| for two normal \lstinline|List|s
  \end{itemize}

  \begin{uncoverenv}<.->
\begin{lstlisting}[style=swift]
func append<A>(_ a: List<A>, _ b: List<A>) -> List<A> {
  switch a {
  case let .Cons(h, t):
    return .Cons(h, append(t, b))
  case .Nil:
    return b
  }
}  
\end{lstlisting}     
  \end{uncoverenv}

  \begin{itemize}
    \item<+-> This \lstinline|append| has two cases depending on
    the first argument
    \begin{itemize}
      \item If it is \lstinline|Cons(h, t)|,
      then the \lstinline|append| calls itself
      passing \lstinline|t| as the first argument recursively
      
      \item If it is \lstinline|Nil|,
      then the function returns the second argument
    \end{itemize}

    \item<+-> This \lstinline|append| is using \emph{run time} dispatching,
    but we want to dispatch in \emph{compile time}
  \end{itemize}

  \begin{notes}
    \item 用語
    \begin{description}
      \item[recursively] 再帰的に
      \item[run time] ランタイム
      \item[compile time] コンパイルタイム
    \end{description}

    \item Transcript:
    もう一度、普通のリストのappendを見てみると、
    これは最初の引数によって2つに場合分けしている。
    もしそれがCons(h, t)ならば、appendを再帰的に呼び出し
    一方でNilならば第二引数をそのまま返す。
    つまりリストの長さによって挙動を変更しているが、
    リストの長さはランタイムの情報である。
    ただ、HListは型が要素の順番に並んでいるので、
    型として長さを持っている。型はコンパイルタイムの情報なので、
    コンパイルタイムに挙動を変更しなければならない。
    ここでextensionを利用する。
  \end{notes}
\end{frame}

\begin{frame}[fragile]
  \frametitle{Implementation}

  \begin{itemize}
    \item<+-> We are making a \lstinline|protocol| for the \lstinline|HList|s \lstinline|append| as follows
  \end{itemize}

  \begin{uncoverenv}<.->
\begin{lstlisting}[style=swift]
protocol HAppend {
  associatedtype Left: HList
  associatedtype Right: HList
  associatedtype Result: HList
   
  static func append(_ l1: Left, _ l2: Right) -> Result
}
\end{lstlisting}
  \end{uncoverenv}

  \begin{itemize}
    \item<+-> Then we'll think about its \lstinline|extension|
  \end{itemize}

  \begin{notes}
    \item Transcript:
    ということで、まずは``HAppend''というHListのappendのprotocolを定義する。
    これはLeftとRightという2つのHListを取り、それを結合した結果の
    ResultというHListを返す関数appendをあらわすprotocolである。
    これのextensionがどうなるかを今後は考えていく。
  \end{notes}
\end{frame}

\begin{frame}[fragile]
  \frametitle{Implementation}

  \begin{itemize}
    \item<+-> First we can create a pettern for that the
    first argument is \lstinline|HNil|
  \end{itemize}

  \begin{uncoverenv}<.->
\begin{lstlisting}[style=swift]
class App1<A: HList> {
  init() {}
}

extension App1: HAppend {
  typealias Left = HNil
  typealias Right = A
  typealias Result = A
   
  static func append(_ l1: HNil, _ l2: A) -> A {
    return l2
  }
}
\end{lstlisting}
  \end{uncoverenv}

  \begin{itemize}
    \item<+-> It just returns the second argument
  \end{itemize}

  \begin{notes}
    \item Transcript:
    ということで、まずは簡単そうなHNilのケースから考えよう。
    HList版appendの最初の引数は、associatedtype Leftで
    与えられるので、それがHNilであった場合を考える。
    これは単に第二引数のHListをそのまま返せばいいが、
    ただここで問題となるのは第二引数を表すHListの型が手元にない。
    extensionが型パラメータを取れたらいいが、それはできないので、
    適当な型App1を作って対処することにする。
  \end{notes}
\end{frame}

\begin{frame}[fragile]
  \frametitle{Implementation}

  \begin{itemize}
    \item<+-> Next we have to make a \lstinline|HCons| pattern
  \end{itemize}

  \begin{uncoverenv}<.->
\begin{lstlisting}[style=swift]
extension App1: HAppend where A == HCons<?, ?> {
  ??? // There is non-working......
}
\end{lstlisting}
  \end{uncoverenv}

  \begin{itemize}
    \item<+-> How do we get the \lstinline|?| types?
    so it doesn't work well...\ce{😇}

    \item<+-> There are two problems
    \begin{enumerate}
      \item A structure (\lstinline|class|, \lstinline|struct| and \lstinline|protcol|)
      can be conformed by the same \lstinline|protocol| \emph{at most one}

      \item And how do we get the \lstinline|append| function
      that can concat previous \lstinline|HList|s?
    \end{enumerate}
  \end{itemize}
  
  \begin{notes}
    \item Transcript:
    ところが次にHConsのパターンを考えるとむつかしくなる。
    まず、App1はさきほどHNilのパターンでHAppendに適合させてしまった。
    したがってもう一度HAppendに適合させることはSwiftが許してくれない。
    また、HCons(h, t)のように分解したとして、
    ではこのtを使ったappendをどこから持ってくればよいだろうか。
    普通のListのappendでは型が同じだったのでよかったが、
    一方でHListではそうはいかない。
  \end{notes}
\end{frame}

\begin{frame}[fragile]
  \frametitle{Solution for the former}

  \begin{itemize}
    \item<+-> We make an adapter class \lstinline|App3| following below
  \end{itemize}

  \begin{uncoverenv}<.->
    \begin{columns}
      \begin{column}{0.43\textwidth}
\begin{lstlisting}[style=swift]
class App3<A: HList, B: HList, C> {
  init() {}
}
\end{lstlisting}
      \end{column}
      \begin{column}{0.57\textwidth}
\begin{lstlisting}[style=swift]
extension App3: HAppend where A == HCons<C, B> {
  ???
}   
\end{lstlisting}
      \end{column}
    \end{columns}
  \end{uncoverenv}

  \begin{itemize}
    \item<.-> We can use \lstinline|App3| for \lstinline|HCons| case instead of \lstinline|App1|
    
    \item<+-> Then we have to decide the \lstinline|associatedtype|s
    of \lstinline|HAppend|
  \end{itemize}

  \begin{uncoverenv}<.->
\begin{lstlisting}[style=swift]
extension App3: HAppend where A == HCons<C, B> {
  typealias Left = A
  typealias Right = ? // What should we put?
  typealias Result = ? // What should we put?
}
\end{lstlisting}
  \end{uncoverenv}

  \uncover<+->{
    \simplecallout{}{red!30}{What should we put into types:
      \lstinline|Right| and \lstinline|Result|?}
  }
  
  \begin{notes}
    \item 用語
    \begin{description}
      \item[App] アップ(Applyの先頭3文字)
    \end{description}

    \item Transcript:
    かなり強引だが、3つ型パラメータを取れるようにしてしまえば、
    HAppendのassociatedtypeであるLeft、Right、Resultに
    対応させることができる。かつApp3はまだHAppendが適合していないので、
    HNilのケースであるApp1と衝突することがない。
    これでとりあえず最初の問題を解決する。
    ただ、そうしたときにHAppendが持つ型LeftとRightにどういった
    型を入れればよいだろうか?
  \end{notes}
\end{frame}

\begin{frame}[fragile]
  \frametitle{Solution for the latter}

  \begin{itemize}
    \item<+-> We add a type parameter called \lstinline|EV|
    to get the previous \lstinline|append|

    \item<+-> For this, we add 2 type parameters then
    the adapter class takes 5 parameters rather than 3
  \end{itemize}

  \begin{uncoverenv}<.->
\begin{lstlisting}[style=swift]
class App5<A: HList, B: HList, C: HList, D, EV: HAppend> {
  init() {}
}
\end{lstlisting}
  \end{uncoverenv}

  \begin{itemize}
    \item<+-> Finally we can substitute the types for \lstinline|associatedtype|s
    declared by \lstinline|HAppend|
  \end{itemize}

  \begin{uncoverenv}<.->
\begin{lstlisting}[style=swift]
extension App5: HAppend where A == EV.Left, B == EV.Right, C == EV.Result {
  typealias Left = HCons<D, A>
  typealias Right = X.B
  typealias Result = HCons<D, C>
}
\end{lstlisting}
  \end{uncoverenv}

  \begin{notes}
    \item 用語
    \begin{description}
      \item[EV] エビデンス(evidence、証拠)
    \end{description}

    \item Transcript:
    ただ、それだけでは1つ前のappend、つまりは
    「1つ小さいHList用のappendを探してくる」ことができない。
    Swiftは型から機能を探してくることができるので、型に機能を割り当てればいい。
    幸いなことにHListのappendは型から機能が唯一に決定できる。
    現在まさに定義しようとしているHAppendのLeftがHCons<H, T>だとしたら、
    1つ前のHAppendはLeftがTである。
    このようなappendを探すために、HListであるA, B, Cを取り、EVという型は
    HListであるAとHListであるBを受け取って結合し、そしてHList Cを返す機能を持つとする。
    そうすれば任意の型Dについて、HCons<D, A>とBを受け取ってHCons<D, C>を返す
    appendを作ることができる。こうしてListのappendようなランタイムの再帰を
    HListの型レベルの再帰に持ち込むことができる。
  \end{notes}
\end{frame}

\begin{frame}[fragile]
  \frametitle{Solution for the latter}

\begin{lstlisting}[style=swift]
extension App5: HAppend where A == EV.Left, B == EV.Right, C == EV.Result {
  typealias Left = HCons<D, A>
  typealias Right = X.B
  typealias Result = HCons<D, C>
}
\end{lstlisting}

  \begin{itemize}
    \item It can be figured out by the picture
  \end{itemize}

  \pause
  \[
    \text{A} := \STwo{?_1, ?_2},\; \text{B} := \STwo{?_3, ?_4},\; \text{C} := \text{A} + \text{B} = \SFour{?_1, ?_2, ?_3, ?_4}
  \]

  \pause
  \begin{align*}
    \text{Left}\; &:= HCons(\mathbf{D}, A) = \SThree{\mathbf{D}, ?_1, ?_2} \\
    \text{Right}\; &:= \text{B} = \STwo{?_3, ?_4} \\
    \text{Result}\; &:= HCons(\mathbf{D}, \text{A}) + \text{B} = \SFive{\mathbf{D}, ?_1, ?_2, ?_3, ?_4} = HCons(D, C)
  \end{align*}
\end{frame}

\begin{frame}[fragile]
  \frametitle{Complete implementation}

  \begin{itemize}
    \item<+-> We get the perfect implementation for \lstinline|HList|s \lstinline|append| function
  \end{itemize}

  \begin{uncoverenv}<.->
\begin{lstlisting}[style=swift]
extension App1: HAppend {
  typealias Left = HNil
  typealias Right = A
  typealias Result = A
   
  static func append(_ l1: HNil, _ l2: A) -> A {
    return l2
  }
}
extension App5: HAppend where EV.Left == A, B == EV.Right, C == EV.Result {
  typealias Left = HCons<D, A>
  typealias Right = B
  typealias Result = HCons<D, C>
   
  static func append(_ l1: HCons<D, A>, _ l2: B) -> HCons<D, C> {
    return HCons(l1.head, EV.append(l1.tail, l2))
  }
}
\end{lstlisting}
  \end{uncoverenv}

  \begin{notes}
    \item Transcript:
    ようやくできあがった。
    これが実用的かどうかは正直なところわからなくなりつつある。
    なぜそう思うのかを説明するために、できあがったappendの
    利用例を見ていこう。
  \end{notes}
\end{frame}

\begin{frame}[fragile]
  \frametitle{Example}

\begin{lstlisting}[style=swift]
let a = HCons(1, HCons("string", HNil.hNil))
let b = HCons(true, HCons(1.001, HNil.hNil))

typealias AA = HCons<String, HNil>
typealias A = HCons<Int, AA>
typealias BB = HCons<Double, HNil>
typealias B = HCons<Bool, BB>
typealias CC = HCons<String, B>
typealias C = HCons<Int, CC>
typealias AppC = App1<B>
typealias AppB = App5<HNil, B, B, String, AppC>
typealias AppA = App5<AA, B, CC, Int, AppB>

let appended = AppA.append(a, b)
\end{lstlisting}

  \uncover<.->{
    \simplecallout{-}{green!30}{This is it! \ce{😇}}
  }

  \begin{notes}
    \item Transcript:
    このようにEVを手で与えなければならないため、
    ちょっと実用で使えるものにはならない……。
  \end{notes}
\end{frame}

\begin{frame}[fragile]
  \frametitle{Example}

  \begin{itemize}
    \item<+-> We have to give Swift what the type \lstinline|EV| is
    because Swift doesn't infer it
  \end{itemize}

  \begin{notes}
    \item 用語
    \begin{description}
      \item[type inference] 型推論(この日本語を言う可能性が高い)
    \end{description}
    
    \item Transcript:
    いろいろな方法を試して、なんとかEVを自動で得ることはできないだろうか?
    ということを考えていたものの、どれも上手くいかなかった。
    たぶんEVは手動で入れるしかないだろう。
  \end{notes}
\end{frame}

\section{Conclusion}

\begin{frame}[fragile]
  \frametitle{Conclusion}

  \pause
  \begin{itemize}
    \item<+-> There are some interesting data types like \lstinline|HList|

    \item<+-> \lstinline|append| method is able to make by
    \lstinline|protocol| and \lstinline|extension|

    \item<+-> Swift's type inference is wanted by people to
    improve itself
    \begin{itemize}
      \item \lstinline|HList|s are very useful
      but we'll not use their \lstinline|append| in the production
    \end{itemize}
  \end{itemize}

  \begin{notes}
    \item Transcript:
    このトークではHListについて紹介し、そしてそれらの
    appendを実装していった。
    さて、もし僕のトークのアブストラクトを読んでいる人がいたら
    「どこにカリーハワード対応がでてきた?」と思うかもしれない。
    実はそういう方法をやってみたものの、
    紹介したApp1やApp5を利用したテクニックでしか、
    Swiftのprotocol/extensionにジェネリクスを持ち込むことができなかった。
    今回、挫折した方法も今後さらに考えていきたいとは思う。
    また、Swiftの型推論がよりよくなれば、もっとよいコードが書けるようになると思う。
  \end{notes}
\end{frame}

\section*{References}

\begin{frame}
  \frametitle{References}

  \bibliographystyle{junsrt_url}
  \nocite{*}
  \bibliography{ref}

  \begin{notes}
    \item Transcript:
    今回のトークでもし型システムに興味を持たれた方は、
    ピアースさんのこの本(上が原著、下は日本語訳)をおすすめしたい。
    型システムの基礎から本格的なところまで詳細に記述してある。
    形式的なところはどうしても「定義・定理・証明」みたいな流れで
    ちょっと困惑するかもしれないが、実はそこまですごく
    最初のハードルが高いわけではないと思う。
    もし興味があるなら買ってみてほしい。
  \end{notes}
\end{frame}

\begin{frame}
  \centering
  {\Huge Thank you for your attention!}

  \begin{notes}
    \item Transcript:
    ご静聴ありがとうございました。
  \end{notes}
\end{frame}

\end{document}
